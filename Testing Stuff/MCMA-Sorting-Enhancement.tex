% Options for packages loaded elsewhere
\PassOptionsToPackage{unicode}{hyperref}
\PassOptionsToPackage{hyphens}{url}
%
\documentclass[
]{article}
\usepackage{lmodern}
\usepackage{amssymb,amsmath}
\usepackage{ifxetex,ifluatex}
\ifnum 0\ifxetex 1\fi\ifluatex 1\fi=0 % if pdftex
  \usepackage[T1]{fontenc}
  \usepackage[utf8]{inputenc}
  \usepackage{textcomp} % provide euro and other symbols
\else % if luatex or xetex
  \usepackage{unicode-math}
  \defaultfontfeatures{Scale=MatchLowercase}
  \defaultfontfeatures[\rmfamily]{Ligatures=TeX,Scale=1}
\fi
% Use upquote if available, for straight quotes in verbatim environments
\IfFileExists{upquote.sty}{\usepackage{upquote}}{}
\IfFileExists{microtype.sty}{% use microtype if available
  \usepackage[]{microtype}
  \UseMicrotypeSet[protrusion]{basicmath} % disable protrusion for tt fonts
}{}
\makeatletter
\@ifundefined{KOMAClassName}{% if non-KOMA class
  \IfFileExists{parskip.sty}{%
    \usepackage{parskip}
  }{% else
    \setlength{\parindent}{0pt}
    \setlength{\parskip}{6pt plus 2pt minus 1pt}}
}{% if KOMA class
  \KOMAoptions{parskip=half}}
\makeatother
\usepackage{xcolor}
\IfFileExists{xurl.sty}{\usepackage{xurl}}{} % add URL line breaks if available
\IfFileExists{bookmark.sty}{\usepackage{bookmark}}{\usepackage{hyperref}}
\hypersetup{
  pdftitle={MCMA Sorting Enhancement},
  pdfauthor={Matthew Hudes},
  hidelinks,
  pdfcreator={LaTeX via pandoc}}
\urlstyle{same} % disable monospaced font for URLs
\usepackage[margin=1in]{geometry}
\usepackage{color}
\usepackage{fancyvrb}
\newcommand{\VerbBar}{|}
\newcommand{\VERB}{\Verb[commandchars=\\\{\}]}
\DefineVerbatimEnvironment{Highlighting}{Verbatim}{commandchars=\\\{\}}
% Add ',fontsize=\small' for more characters per line
\usepackage{framed}
\definecolor{shadecolor}{RGB}{248,248,248}
\newenvironment{Shaded}{\begin{snugshade}}{\end{snugshade}}
\newcommand{\AlertTok}[1]{\textcolor[rgb]{0.94,0.16,0.16}{#1}}
\newcommand{\AnnotationTok}[1]{\textcolor[rgb]{0.56,0.35,0.01}{\textbf{\textit{#1}}}}
\newcommand{\AttributeTok}[1]{\textcolor[rgb]{0.77,0.63,0.00}{#1}}
\newcommand{\BaseNTok}[1]{\textcolor[rgb]{0.00,0.00,0.81}{#1}}
\newcommand{\BuiltInTok}[1]{#1}
\newcommand{\CharTok}[1]{\textcolor[rgb]{0.31,0.60,0.02}{#1}}
\newcommand{\CommentTok}[1]{\textcolor[rgb]{0.56,0.35,0.01}{\textit{#1}}}
\newcommand{\CommentVarTok}[1]{\textcolor[rgb]{0.56,0.35,0.01}{\textbf{\textit{#1}}}}
\newcommand{\ConstantTok}[1]{\textcolor[rgb]{0.00,0.00,0.00}{#1}}
\newcommand{\ControlFlowTok}[1]{\textcolor[rgb]{0.13,0.29,0.53}{\textbf{#1}}}
\newcommand{\DataTypeTok}[1]{\textcolor[rgb]{0.13,0.29,0.53}{#1}}
\newcommand{\DecValTok}[1]{\textcolor[rgb]{0.00,0.00,0.81}{#1}}
\newcommand{\DocumentationTok}[1]{\textcolor[rgb]{0.56,0.35,0.01}{\textbf{\textit{#1}}}}
\newcommand{\ErrorTok}[1]{\textcolor[rgb]{0.64,0.00,0.00}{\textbf{#1}}}
\newcommand{\ExtensionTok}[1]{#1}
\newcommand{\FloatTok}[1]{\textcolor[rgb]{0.00,0.00,0.81}{#1}}
\newcommand{\FunctionTok}[1]{\textcolor[rgb]{0.00,0.00,0.00}{#1}}
\newcommand{\ImportTok}[1]{#1}
\newcommand{\InformationTok}[1]{\textcolor[rgb]{0.56,0.35,0.01}{\textbf{\textit{#1}}}}
\newcommand{\KeywordTok}[1]{\textcolor[rgb]{0.13,0.29,0.53}{\textbf{#1}}}
\newcommand{\NormalTok}[1]{#1}
\newcommand{\OperatorTok}[1]{\textcolor[rgb]{0.81,0.36,0.00}{\textbf{#1}}}
\newcommand{\OtherTok}[1]{\textcolor[rgb]{0.56,0.35,0.01}{#1}}
\newcommand{\PreprocessorTok}[1]{\textcolor[rgb]{0.56,0.35,0.01}{\textit{#1}}}
\newcommand{\RegionMarkerTok}[1]{#1}
\newcommand{\SpecialCharTok}[1]{\textcolor[rgb]{0.00,0.00,0.00}{#1}}
\newcommand{\SpecialStringTok}[1]{\textcolor[rgb]{0.31,0.60,0.02}{#1}}
\newcommand{\StringTok}[1]{\textcolor[rgb]{0.31,0.60,0.02}{#1}}
\newcommand{\VariableTok}[1]{\textcolor[rgb]{0.00,0.00,0.00}{#1}}
\newcommand{\VerbatimStringTok}[1]{\textcolor[rgb]{0.31,0.60,0.02}{#1}}
\newcommand{\WarningTok}[1]{\textcolor[rgb]{0.56,0.35,0.01}{\textbf{\textit{#1}}}}
\usepackage{longtable,booktabs}
% Correct order of tables after \paragraph or \subparagraph
\usepackage{etoolbox}
\makeatletter
\patchcmd\longtable{\par}{\if@noskipsec\mbox{}\fi\par}{}{}
\makeatother
% Allow footnotes in longtable head/foot
\IfFileExists{footnotehyper.sty}{\usepackage{footnotehyper}}{\usepackage{footnote}}
\makesavenoteenv{longtable}
\usepackage{graphicx}
\makeatletter
\def\maxwidth{\ifdim\Gin@nat@width>\linewidth\linewidth\else\Gin@nat@width\fi}
\def\maxheight{\ifdim\Gin@nat@height>\textheight\textheight\else\Gin@nat@height\fi}
\makeatother
% Scale images if necessary, so that they will not overflow the page
% margins by default, and it is still possible to overwrite the defaults
% using explicit options in \includegraphics[width, height, ...]{}
\setkeys{Gin}{width=\maxwidth,height=\maxheight,keepaspectratio}
% Set default figure placement to htbp
\makeatletter
\def\fps@figure{htbp}
\makeatother
\setlength{\emergencystretch}{3em} % prevent overfull lines
\providecommand{\tightlist}{%
  \setlength{\itemsep}{0pt}\setlength{\parskip}{0pt}}
\setcounter{secnumdepth}{-\maxdimen} % remove section numbering

\title{MCMA Sorting Enhancement}
\author{Matthew Hudes}
\date{4/7/2020}

\begin{document}
\maketitle

\hypertarget{hello-in-this-document-we-will-go-over-the-enhancement-made-that-sorts-mcma-type-questions}{%
\subsection{Hello! In this document we will go over the enhancement made
that sorts MCMA type
questions}\label{hello-in-this-document-we-will-go-over-the-enhancement-made-that-sorts-mcma-type-questions}}

\hypertarget{first-we-load-the-package-then-we-call-get-setup-with-our-example-survey}{%
\subsubsection{First we load the package, then we call get setup with
our example
survey}\label{first-we-load-the-package-then-we-call-get-setup-with-our-example-survey}}

\begin{Shaded}
\begin{Highlighting}[]
\NormalTok{devtools}\OperatorTok{::}\KeywordTok{load\_all}\NormalTok{(}\StringTok{"."}\NormalTok{)}
\end{Highlighting}
\end{Shaded}

\begin{verbatim}
## Loading QualtricsTools
\end{verbatim}

\begin{verbatim}
## Warning: package 'testthat' was built under R version 3.6.3
\end{verbatim}

\begin{Shaded}
\begin{Highlighting}[]
\KeywordTok{get\_setup}\NormalTok{(}
  \DataTypeTok{qsf\_path =}\NormalTok{ here}\OperatorTok{::}\KeywordTok{here}\NormalTok{(}\StringTok{"data"}\NormalTok{, }\StringTok{"Sample Surveys"}\NormalTok{, }\StringTok{"Dummy Enhancement Sample Survey"}\NormalTok{, }\StringTok{"Dummy\_Enhancement\_Sample\_Survey.qsf"}\NormalTok{),}
  \DataTypeTok{csv\_path =}\NormalTok{ here}\OperatorTok{::}\KeywordTok{here}\NormalTok{(}\StringTok{"data"}\NormalTok{, }\StringTok{"Sample Surveys"}\NormalTok{, }\StringTok{"Dummy Enhancement Sample Survey"}\NormalTok{, }\StringTok{"Dummy\_Enhancement\_Sample\_Survey.csv"}\NormalTok{),}
  \DataTypeTok{headerrows =} \DecValTok{3}
\NormalTok{)}
\end{Highlighting}
\end{Shaded}

\begin{verbatim}
## survey, responses, questions, blocks, original_first_rows,
##         and flow have now been made global objects.
\end{verbatim}

\hypertarget{so-lets-grab-the-10th-question-from-the-survey---it-is-mcma-without-any-recode-values.}{%
\subsubsection{So lets grab the 10th question from the survey - it is
MCMA without any recode
values.}\label{so-lets-grab-the-10th-question-from-the-survey---it-is-mcma-without-any-recode-values.}}

\begin{Shaded}
\begin{Highlighting}[]
\NormalTok{question \textless{}{-}}\StringTok{ }\NormalTok{questions[[}\DecValTok{10}\NormalTok{]]}
\end{Highlighting}
\end{Shaded}

\hypertarget{now-lets-generate-a-table-with-this-question.-the-table-by-default-will-be-sorted-by-the-n-column}{%
\subsubsection{Now lets generate a Table with this question. The table
by default will be sorted by the N
column}\label{now-lets-generate-a-table-with-this-question.-the-table-by-default-will-be-sorted-by-the-n-column}}

\begin{Shaded}
\begin{Highlighting}[]
\NormalTok{question \textless{}{-}}\StringTok{ }\KeywordTok{mc\_multiple\_answer\_results}\NormalTok{(question, original\_first\_rows)}

\CommentTok{\#\# This code prints that table out:}
\NormalTok{knitr}\OperatorTok{::}\KeywordTok{kable}\NormalTok{(question}\OperatorTok{$}\NormalTok{Table)}
\end{Highlighting}
\end{Shaded}

\begin{longtable}[]{@{}rll@{}}
\toprule
N & Percent &\tabularnewline
\midrule
\endhead
34 & 68.0\% & NA, I don't have any opinion.\tabularnewline
34 & 68.0\% & Neither agree nor disagree\tabularnewline
31 & 62.0\% & Agree\tabularnewline
31 & 62.0\% & Somewhat agree\tabularnewline
31 & 62.0\% & Strongly agree\tabularnewline
31 & 62.0\% & Strongly disagree\tabularnewline
29 & 58.0\% & NA, I haven't experienced this.\tabularnewline
27 & 54.0\% & Somewhat disagree\tabularnewline
25 & 50.0\% & Disagree\tabularnewline
23 & 46.0\% & NA, I really just don't care\ldots.\tabularnewline
\bottomrule
\end{longtable}

\end{document}
